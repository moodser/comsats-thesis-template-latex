\usepackage{amsmath} %Never write a paper without using amsmath for its many new commands
\usepackage{amssymb} %Some extra symbols
\usepackage{makeidx} %If you want to generate an index, automatically
\usepackage{graphicx} %If you want to include postscript graphics
\usepackage{amsfonts}
\usepackage{newlfont}
\PassOptionsToPackage{table,xcdraw}{xcolor}
\usepackage[table, dvipsnames]{xcolor}
\usepackage{ragged2e}
\usepackage{psfrag}
\usepackage{epsfig}
\usepackage{color}
\usepackage{epstopdf}

\usepackage{titlesec}
\usepackage{lipsum}

\usepackage{longtable}
\usepackage{caption}
\usepackage{tabularx}

\usepackage{setspace}

\usepackage[scale=0.74,top=3cm, left=3.3cm]{geometry}

\makeindex



\newtheorem{definition}{Definition}[section]
\newtheorem{proposition}[definition]{Proposition}
\newtheorem{lemma}[definition]{Lemma}
\newtheorem{theorem}[definition]{Theorem}
\newtheorem{corollary}[definition]{Corollary}
\newtheorem{remark}[definition]{Remark}
\newtheorem{observation}[definition]{Observation}

\newcommand{\Buchi}{B$\buildrel {..}\over{\rm u}$chi }
\newcommand{\EhFr}{Ehrenfeucht-Fra\"{\i}ss\'{e} }
\newcommand{\Boj}{Boja\'{n}czyk}
\newcommand{\TLEFs} {$\sf{TL[EF_s]}\mbox{ }$}
\newcommand{\TLEF} {$\sf{TL[EF]}\mbox{ }$}

\newcommand{\qed}{\hfill $\Box$ \hfill \\}

\renewcommand{\labelenumi}{(\roman{enumi})}

%\renewcommand{\baselinestretch}{1.66}



\usepackage{arabtex,lipsum}

%%% from Uxnsh.fd
\DeclareFontFamily{U}{xnsh}{}%

\DeclareFontShape{U}{xnsh}{m}{n}{%                                   
	<-6> sfixed * [6.0] xnsh14
	<6-10> s * [1.20] xnsh14
	<10><10.95><12><14.4><17.28><20.74><24.88> s * [1.20] xnsh14
}{}

\DeclareFontShape{U}{xnsh}{bx}{n}{%
	<-6> sfixed * [6.0] xnsh14bf
	<6-10> s * [1.20] xnsh14bf
	<10><10.95><12><14.4><17.28><20.74><24.88> s * [1.20] xnsh14bf
}{}
%%% end of added code

\usepackage{utf8}
\setcode{utf8}


\renewcommand{\baselinestretch}{1} % line for space in paragraphs

\newcommand\tab[1][0.5cm]{\hspace*{#1}}


\usepackage[hidelinks]{hyperref}



